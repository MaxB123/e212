\documentclass[10pt, a4paper, notitlepage, DIV=15]{scrartcl}
\usepackage[ngerman, english]{babel}
\usepackage[utf8]{inputenc}
\usepackage[T1]{fontenc}
\usepackage{csquotes}
\usepackage{xcolor}
\usepackage{amsmath}
\usepackage{amsfonts}
\usepackage{graphicx}
\usepackage{siunitx}
\usepackage{csquotes}
\usepackage{hyperref}


\title{Properties of Elementary Particles}
\author{Max Bock \\ Email \href{mailto:s6mabock@uni-bonn.de}{s6mabock@uni-bonn.de} 
	\and Marvin Hoffmann \\ Email \href{mailto:marvin.hoffmann@uni-bonn.de}{marvin.hoffmann@uni-bonn.de} }

%Nature has always looked like a horrible mess, but as we go along we see patterns and put theories together; a certain clarity comes and things get simpler. (Richard P. Feynman (R.P Feynam: QED- The Strange Theory of Light and Matter (Princeton UNiversity Press, Princeton 1985)))%


\begin{document}
\maketitle
\tableofcontents
\newpage
\section{Introduction}

\section{Theoretical Background}
\subsection{Basics of particle physics}
\subsubsection{Basics of Standard Model of Particle Physics}
The Standard Model of particle physics implies the unified theory of the electromagnetic, weak and strong interaction and decides the elementary particles into two classes, namely bosons and fermions. \newline
The bosons acts as exchange particles of the fundamental forces and have an integer spin. These ones are often called gauge bosons. There are eight massless gluons, which carry the colour charge of the strong force. Due to the fact that the gluons themselves have a colour charge, they interact with each other. The interaction range of the gluon is the smallest of all gauge bosons and is in the order of $1\,$fm. The massless photon is the exchange particle of the electromagnetic force, which has no electromagnetic charge. The range of this gauge boson is infinity. For the weak interaction there are two exchange particles, the $Z$ boson which carries no electromagnetic charge, whereby the $W^{\pm}$ bosons have a charge of $\pm 1$. The last boson is the so called Higgs particle- This is a scalar boson which delivers the mass of all particles which are coupled to the Higgs field.  \newline
The fermions are divided into two subclasses, the quarks and the leptons which are also subdivided into three generations. The quarks are the only particles which interact strongly and carry colour charge but also interacts electromagnetic and weak. Due to the fact that the quarks can not appear alone apart they must form colour neutral formations. These formations are called hadrons which are divided into two classes, the mesons and the baryons. The mesons consist of a quark-antiquark pair and the baryons consist of three quarks. Each quark has either a charge of $2/3$ or $-1/3$ and further a spin of $1/2$. 
\newline
The leptons consist of electron, muon and tau with the corresponding neutrino.  
\begin{figure}[h]
	\centering
	\includegraphics[width=0.6\textwidth]{Standard_model.png}
	\caption{}
	\label{fig:standard_model}
\end{figure}
\subsection{Bubble Chamber}
\subsection{Proton Proton interaction}
\subsection{$\omega$ Meson}

\section{Experimental Set-Up and Measurements}

\section{Analysis}

\section{Conclusion}

\section{Appendix}

\begin{thebibliography}{}
\bibitem{lasers}
	Peter W. Milonni (Los Alamos National Laboratory) and Joseph H. Eberly (University of Rochester), \textit{Lasers} (Wiley-Interscience, 1988)
\bibitem{laserphysics}
	Simon Hooker and Colin Webb, Department of Physics, University of Oxford, \textit{Laser Physics}, (Oxford University Press, 2011)
\bibitem{quantumefficiency}
Massachusetts Institute of Technology (MIT), Electrical Engineering and Computer Science, Compound Semiconductor Devices, 2003 \url{https://ocw.mit.edu/courses/electrical-engineering-and-computer-science/6-772-compound-semiconductor-devices-spring-2003/lecture-notes/lecture20.pdf}
\bibitem{bergmann}
	Bergmann and Schäfer, \textit{Lehrbuch der Experimentalphysik, Band 3, Optik und Wellenmechanik}, de Gruyter Verlag, 10th. edition (2004)
\bibitem{boyd}
	R. W. Boyd, \textit{Nonlinear Optics}, Academic Press Boston (1992)
\bibitem{meschede}
D. Meschede, \textit{Optik, Licht und Laser}, Vieweg+Teubner Verlag, 3.rd edition (2008)
\bibitem{hecht}
	E. Hecht, \textit{Optics}, Addison Wesley, 4.th edition (2002)
\bibitem{kühlke}
	D. Kühlke, \textit{Optik - Grundlagen und Anwendungen}, Verlag Harri Deutsch, 3.rd improved edition (2011)
\bibitem{wiki-gauss}
	\url{https://en.wikipedia.org/wiki/Gaussian_beam#/media/File:GaussianBeamWaist.svg}
\bibitem{description}
	Experimental Description, \textit{A245: Optical Frequency Doubling}, University of Bonn, August 2015
\bibitem{zinth}
	W. Zinth and U. Zinth, \textit{Optik} \textit{Lichtstrahlen-Wellen-Photonen}, Oldenbourg Verlag München, 4th. improved edition (2013)
	

\end{thebibliography}
 
\end{document}